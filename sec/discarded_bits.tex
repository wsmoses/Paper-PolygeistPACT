%%%
%%% This file contains bits discarded from the main file, but potentially useful in the future.
%%% We are reinventing version control... They would have been recoverable from git with proper version control.
%%%

Practically, MLIR defines an SSA-based~\cite{ssa} IR and provides algorithms and tools to analyze and transform it.
Like many SSA representations, MLIR uses \emph{values} as a unit of data processed by the represented program.
MLIR values cannot be redefined. \emph{Operations} describe the semantic of a program. Operations are a generalization of (machine) instructions or high-abstraction operations such as \texttt{matmul}, that use values and define new values.
Operations are the primary mechanism for defining the semantics of the program.
Values in MLIR have a \emph{type}, which contains the information known about the \emph{value} at compile time.
Similarly, operation \emph{attributes} contain the additional information about the operation known at compile time.
Operations are organized into linear sequences of (basic) \emph{blocks} that are executed sequentially.
Blocks may accept values as arguments, following the functional SSA form~\cite{appel1998ssa} as opposed to the $\phi$-node form.
\emph{Regions} contain groups of blocks.
MLIR supports regions with classical control flow graph (CFG) structure where the control can flow from one block to one of its \emph{successors} and arbitrary graph regions that can have custom semantics such as TensorFlow graphs.
We only consider CFG regions throughout this paper.
Regions can be attached to an operation, which defines how the control flows into and from these regions, allowing the IR to be arbitrarily nested at multiple levels.

%%%%%%%%%%%%%%%%%%%%%%

Some operators within MLIR may not directly correspond to C and C++-level semantics. For example, high-level loop operators within MLIR do not permit the use of the \texttt{continue} construct. When this abstraction mismatch occurs, \tool emits wrapper code that enables the correct behavior. For example, \texttt{continue} is modeled by creating a new boolean that represents whether a continue statement has been executed and wrapping all loop statements within a conditional that checks whether one hit a \texttt{continue} (Figure \ref{fig:continue}).

\begin{figure}
    \centering
\begin{tabular}{c}
\begin{minipage}[t]{0.95\linewidth}
\begin{minted}[fontsize=\small, escapeinside=||]{cpp}
for (int i=0; i<N; ++i) {
    A(i);
    if (B(i))
        continue;
    C(i);
}
\end{minted}
\end{minipage}\\
\begin{minipage}[t]{0.95\linewidth}
\begin{minted}[fontsize=\small, escapeinside=@@]{cpp}
             @$\big\Downarrow$@
bool seenContinue;
for (int i=0; i<N; ++i) {
    seenContinue = false;
    if (!seenContinue)
        A(i);
    if (!seenContinue)
        if (B(i))
            seenContinue = true;
    if (!seenContinue)
        C(i);
}
\end{minted}
\end{minipage} 
\end{tabular}
    \caption{Top: a C program that contains a continue statement. Bottom: a rewrite of the code above that eliminates the continue statement while preserving the original program's semantics. \tool performs such a transformation when lowering to MLIR (rather than emitting C) as MLIR's high-level loop constructs do not permit \texttt{continue}.}
    \label{fig:continue}
\end{figure}

%===============WIP
\begin{figure*}
\centering
\begin{tikzpicture}
\begin{axis}[
    width=1\textwidth,
    every axis plot post/.style={/pgf/number format/fixed},
    ybar=1pt,
    enlargelimits=0.028,
    ylabel={time},
    legend style={draw=none, fill=none},
    bar width=2pt,
    legend columns=20,
    ymode=log,
    log basis y={10},
    ymin=0, ymax=100,
    log origin=infty, % start from the bottom.
    height=0.3\textwidth,
    ymajorgrids=true,
    grid style=dashed,
    log ticks with fixed point,
    axis x line*=bottom,
    x tick label style={xshift=.4em,rotate=45,anchor=east},
    ytick={1, 16, 128},
    yminorticks=true,
    legend style={at={(1,1.2)},anchor=north east},
    xlabel={},
    extra y ticks={1, 4, 16, 32, 64, 128}, % add extra tick
    extra y tick labels={}, % do not print extra tick
    symbolic x coords={gemm, gemver, gesummv, symm, syr2k, syrk, trmm, 2mm, 3mm, atax, bicg, doitgen, mvt},
    xtick=data,
]

% clangsing    
\addplot+ [orange3] coordinates {
(gemm,8.35371)
(gemver,0.159616)
(gesummv,0.013066)
(symm,53.8629)
(syr2k,70.1443)
(syrk,25.9289)
(trmm,47.9134)
(2mm,62.4948)
(3mm,104.73)
(atax,0.0068922)
(bicg,0.0055652)
(doitgen,5.01112)
(mvt,0.143938)
};

% Clang
\addplot+ [blind_safe_three_scheme_four_colors] coordinates {
(gemm,8.46626)
(gemver,0.158418)
(gesummv,0.013282)
(symm,53.7901)
(syr2k,70.3723)
(syrk,25.8021)
(trmm,48.3674)
(2mm,61.9698)
(3mm,104.193)
(atax,0.0067884)
(bicg,0.0055274)
(doitgen,4.91817)
(mvt,0.143474) 
};

% mlir-clang
\addplot+ [blind_safe_four_scheme_four_colors] coordinates {
(gemm,6.94101)
(gemver,0.159908)
(gesummv,0.0133888)
(symm,54.3839)
(syr2k,71.7658)
(syrk,25.9046)
(trmm,48.2991)
(2mm,62.0393)
(3mm,105.27)
(atax,0.0065358)
(bicg,0.0056104)
(doitgen,4.85439)
(mvt,0.144069) 
};

% Polly 
\addplot+ [blind_safe_one_scheme_four_colors] coordinates {
(gemm,10.1609)
(gemver,0.080229)
(gesummv,0.0132624)
(symm,53.7669)
(syr2k,10.0732)
(syrk,5.36067)
(trmm,15.1114)
(2mm,10.7852)
(3mm,17.3721)
(atax,0.0067856)
(bicg,0.0043138)
(doitgen,4.84958)
(mvt,0.0500774) 
};

% Polly par
\addplot+ [blind_safe_two_scheme_four_colors] coordinates {
(gemm,1.33031)
(gemver,0.0202084)
(gesummv,0.0262504)
(symm,53.6737)
(syr2k,2.08874)
(syrk,1.34801)
(trmm,1.96776)
(2mm,1.52907)
(3mm,2.4513)
(atax,0.0168894)
(bicg,0.0092572)
(doitgen,4.92501)
(mvt,0.011839)
};

% Pluto
\addplot+ [blind_safe_two_scheme_four_colors] coordinates {
(gemm,4.59185)
(gemver,0.100849)
(gesummv,0.033856)
(symm,53.7301)
(syr2k,9.99791)
(syrk,5.3404)
(trmm,2.33019)
(2mm,4.36668)
(3mm,7.87485)
(atax,0.0104202)
(bicg,0.0093366)
(doitgen,1.87225)
(mvt,0.0834414)
};

% Pluto par
\addplot+ [blind_safe_two_scheme_four_colors] coordinates {
(gemm,0.6995)
(gemver,0.0281192)
(gesummv,0.0098276)
(symm,53.8059)
(syr2k,2.24339)
(syrk,1.4506)
(trmm,0.373893)
(2mm,1.00752)
(3mm,1.50406)
(atax,0.0053492)
(bicg,0.00517)
(doitgen,1.66634)
(mvt,0.0210356)
};

% Polymer
\addplot+ [blind_safe_two_scheme_four_colors] coordinates {
(gemm,3.70353)
(gemver,0.0660196)
(gesummv,0.033392)
(symm,54.0499)
(syr2k,9.41316)
(syrk,6.06907)
(trmm,1.77204)
(2mm,3.13015)
(3mm,6.64746)
(atax,0.011067)
(bicg,0.0116124)
(doitgen,1.07443)
(mvt,0.0805746)
};

% Polymer par 
\addplot+ [blind_safe_two_scheme_four_colors] coordinates {
(gemm,0.550249)
(gemver,0.0231254)
(gesummv,0.0091832)
(symm,7.00879)
(syr2k,2.06038)
(syrk,1.35607)
(trmm,0.267544)
(2mm,0.843202)
(3mm,1.24121)
(atax,0.0040444)
(bicg,0.0041572)
(doitgen,1.79475)
(mvt,0.0172978)
};



\legend{clangsing, Clang, mlir-clang, polly, pollypar, pluto, plutopar, polymer, polymerpar}
\end{axis}
\end{tikzpicture}
\end{figure*}

\iffalse
\begin{figure}
\centering
\begin{tikzpicture}[auto, node distance=.51cm and 0.6cm]

    \node [process] (clang1) {C Code};

    \node [process, below= of clang1] (opt1) {Clang AST};

    \node [process, below= of opt1] (enzyme1) {MLIR};

    \node [process, below= of enzyme1] (popt1) {Raise Affine};

    
    \node [process, below= of popt1] (clang2) {Polyhedral};

    \node [process, below= of clang2] (enzyme2) {Lower Affine};

    \node [process, below= of enzyme2] (opt2) {Lower SCF};

    \node [process, below= of opt2] (popt2) {Lower LLVM};

    \node [process, below= of popt2] (cg2) {LLVM \icode{opt -O3}};

    \node [process, below= of cg2] (cg1) {Binary};

    \draw [->, line width=0.5mm] (clang1) -- node {} (opt1);
    \draw [->, line width=0.5mm] (opt1) -- node {} (enzyme1);
    \draw [->, line width=0.5mm] (enzyme1) -- node {} (popt1);
    \draw [->, line width=0.5mm] (popt1) -- node {} (clang2);

    \draw [->, line width=0.5mm] (clang2) -- node {} (enzyme2);
    \draw [->, line width=0.5mm] (enzyme2) -- node {} (opt2);
    \draw [->, line width=0.5mm] (opt2) -- node {} (popt2);
    \draw [->, line width=0.5mm] (popt2) -- node {} (cg2);
    \draw [->, line width=0.5mm] (cg2) -- node {} (cg1);
  \end{tikzpicture}
\caption{Pipeline}
\label{fig:pipeline}
\end{figure}
\fi